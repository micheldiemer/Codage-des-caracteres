% Options for packages loaded elsewhere
\PassOptionsToPackage{unicode}{hyperref}
\PassOptionsToPackage{hyphens}{url}
%
\documentclass[
  a4paper,
]{book}
\usepackage{amsmath,amssymb}
\usepackage[]{STIXGeneral}
\usepackage{iftex}
\ifPDFTeX
  \usepackage[T1]{fontenc}
  \usepackage[utf8]{inputenc}
  \usepackage{textcomp} % provide euro and other symbols
\else % if luatex or xetex
  \usepackage{unicode-math}
  \defaultfontfeatures{Scale=MatchLowercase}
  \defaultfontfeatures[\rmfamily]{Ligatures=TeX,Scale=1}
\fi
% Use upquote if available, for straight quotes in verbatim environments
\IfFileExists{upquote.sty}{\usepackage{upquote}}{}
\IfFileExists{microtype.sty}{% use microtype if available
  \usepackage[]{microtype}
  \UseMicrotypeSet[protrusion]{basicmath} % disable protrusion for tt fonts
}{}
\makeatletter
\@ifundefined{KOMAClassName}{% if non-KOMA class
  \IfFileExists{parskip.sty}{%
    \usepackage{parskip}
  }{% else
    \setlength{\parindent}{0pt}
    \setlength{\parskip}{6pt plus 2pt minus 1pt}}
}{% if KOMA class
  \KOMAoptions{parskip=half}}
\makeatother
\usepackage{xcolor}
\usepackage[margin=1in]{geometry}
\usepackage{color}
\usepackage{fancyvrb}
\newcommand{\VerbBar}{|}
\newcommand{\VERB}{\Verb[commandchars=\\\{\}]}
\DefineVerbatimEnvironment{Highlighting}{Verbatim}{commandchars=\\\{\}}
% Add ',fontsize=\small' for more characters per line
\newenvironment{Shaded}{}{}
\newcommand{\AlertTok}[1]{\textcolor[rgb]{1.00,0.00,0.00}{\textbf{#1}}}
\newcommand{\AnnotationTok}[1]{\textcolor[rgb]{0.38,0.63,0.69}{\textbf{\textit{#1}}}}
\newcommand{\AttributeTok}[1]{\textcolor[rgb]{0.49,0.56,0.16}{#1}}
\newcommand{\BaseNTok}[1]{\textcolor[rgb]{0.25,0.63,0.44}{#1}}
\newcommand{\BuiltInTok}[1]{#1}
\newcommand{\CharTok}[1]{\textcolor[rgb]{0.25,0.44,0.63}{#1}}
\newcommand{\CommentTok}[1]{\textcolor[rgb]{0.38,0.63,0.69}{\textit{#1}}}
\newcommand{\CommentVarTok}[1]{\textcolor[rgb]{0.38,0.63,0.69}{\textbf{\textit{#1}}}}
\newcommand{\ConstantTok}[1]{\textcolor[rgb]{0.53,0.00,0.00}{#1}}
\newcommand{\ControlFlowTok}[1]{\textcolor[rgb]{0.00,0.44,0.13}{\textbf{#1}}}
\newcommand{\DataTypeTok}[1]{\textcolor[rgb]{0.56,0.13,0.00}{#1}}
\newcommand{\DecValTok}[1]{\textcolor[rgb]{0.25,0.63,0.44}{#1}}
\newcommand{\DocumentationTok}[1]{\textcolor[rgb]{0.73,0.13,0.13}{\textit{#1}}}
\newcommand{\ErrorTok}[1]{\textcolor[rgb]{1.00,0.00,0.00}{\textbf{#1}}}
\newcommand{\ExtensionTok}[1]{#1}
\newcommand{\FloatTok}[1]{\textcolor[rgb]{0.25,0.63,0.44}{#1}}
\newcommand{\FunctionTok}[1]{\textcolor[rgb]{0.02,0.16,0.49}{#1}}
\newcommand{\ImportTok}[1]{#1}
\newcommand{\InformationTok}[1]{\textcolor[rgb]{0.38,0.63,0.69}{\textbf{\textit{#1}}}}
\newcommand{\KeywordTok}[1]{\textcolor[rgb]{0.00,0.44,0.13}{\textbf{#1}}}
\newcommand{\NormalTok}[1]{#1}
\newcommand{\OperatorTok}[1]{\textcolor[rgb]{0.40,0.40,0.40}{#1}}
\newcommand{\OtherTok}[1]{\textcolor[rgb]{0.00,0.44,0.13}{#1}}
\newcommand{\PreprocessorTok}[1]{\textcolor[rgb]{0.74,0.48,0.00}{#1}}
\newcommand{\RegionMarkerTok}[1]{#1}
\newcommand{\SpecialCharTok}[1]{\textcolor[rgb]{0.25,0.44,0.63}{#1}}
\newcommand{\SpecialStringTok}[1]{\textcolor[rgb]{0.73,0.40,0.53}{#1}}
\newcommand{\StringTok}[1]{\textcolor[rgb]{0.25,0.44,0.63}{#1}}
\newcommand{\VariableTok}[1]{\textcolor[rgb]{0.10,0.09,0.49}{#1}}
\newcommand{\VerbatimStringTok}[1]{\textcolor[rgb]{0.25,0.44,0.63}{#1}}
\newcommand{\WarningTok}[1]{\textcolor[rgb]{0.38,0.63,0.69}{\textbf{\textit{#1}}}}
\usepackage{longtable,booktabs,array}
\usepackage{calc} % for calculating minipage widths
% Correct order of tables after \paragraph or \subparagraph
\usepackage{etoolbox}
\makeatletter
\patchcmd\longtable{\par}{\if@noskipsec\mbox{}\fi\par}{}{}
\makeatother
% Allow footnotes in longtable head/foot
\IfFileExists{footnotehyper.sty}{\usepackage{footnotehyper}}{\usepackage{footnote}}
\makesavenoteenv{longtable}
\setlength{\emergencystretch}{3em} % prevent overfull lines
\providecommand{\tightlist}{%
  \setlength{\itemsep}{0pt}\setlength{\parskip}{0pt}}
\setcounter{secnumdepth}{-\maxdimen} % remove section numbering
\ifLuaTeX
\usepackage[bidi=basic]{babel}
\else
\usepackage[bidi=default]{babel}
\fi
\babelprovide[main,import]{french}
% get rid of language-specific shorthands (see #6817):
\let\LanguageShortHands\languageshorthands
\def\languageshorthands#1{}
\ifLuaTeX
  \usepackage{selnolig}  % disable illegal ligatures
\fi
\IfFileExists{bookmark.sty}{\usepackage{bookmark}}{\usepackage{hyperref}}
\IfFileExists{xurl.sty}{\usepackage{xurl}}{} % add URL line breaks if available
\urlstyle{same} % disable monospaced font for URLs
\hypersetup{
  pdftitle={Typographie},
  pdfauthor={Michel Diemer},
  pdflang={fr-FR},
  pdfkeywords={Unicode, Tiret, Espace, Caractère, Émoji},
  hidelinks,
  pdfcreator={LaTeX via pandoc}}
\fontfamily{Consolas}
\title{Typographie}
\author{Michel Diemer}
\date{2022-07-10}

\begin{document}
\frontmatter
\maketitle

{
\setcounter{tocdepth}{3}
\tableofcontents
}
\mainmatter
\hypertarget{typographie}{%
\section{Typographie}\label{typographie}}

\hypertarget{lettres-de-la-langue-franuxe7aise}{%
\subsection{Lettres de la langue française}\label{lettres-de-la-langue-franuxe7aise}}

\texttt{ABCDEFGHIJKLMNOPQRSTUVWXYZ\ ÀÂÄ\ Ç\ ÉÈÊË\ ÎÏ\ ÔÖ\ ÙÛÜ\ Ÿ\ Æ\ Œ\ espace\ -\ \textquotesingle{}}\strut \\
\texttt{abcdefghijklmnopqrstuvwxyz\ àâä\ ç\ éèêë\ îï\ ôö\ ùûü\ ÿ\ æ\ œ\ espace\ -\ \textquotesingle{}}\strut \\
\texttt{pas\ d\textquotesingle{}espace\ ni\ avant\ ni\ après\ -}\strut \\
\texttt{pas\ d\textquotesingle{}espace\ multiples}\strut \\
Cf. \href{https://xml.insee.fr/schema/commun.html\#ChaineFrancaisOfficielMajuscule_stype}{Insee ChaineFrancaisOfficielMajuscule}\\
Nb : le \texttt{ñ\ Ñ} ne fait pas officiellement partie des signes diacritiques de la langue française.\\
Il n'est plus autorisé dans les noms de famille.

\hypertarget{typographie-langue-franuxe7aise}{%
\subsection{Typographie langue française}\label{typographie-langue-franuxe7aise}}

\texttt{symboles\ de\ ponctuation\ ,\ ;\ :\ .\ …\ ?\ !\ «\ »\ (\ )\ {[}\ {]}\ *\ /\ -\ ’}\strut \\
\texttt{trait\ d\textquotesingle{}union\ -\ tiret\ demi-cadratin\ –\ tiret\ cadratin\ —}\strut \\
\texttt{trait\ d\textquotesingle{}union\ conditionnel\ -\ U+00AD\ :\ «\ malaise\ »\ se\ coupe\ en\ mal-aise\ (nm)\ ou\ en\ ma-laise\ (nf)}\strut \\
\texttt{trait\ d\textquotesingle{}union\ insécable\ -\ U+2011\ ‑} \texttt{espace\ espace\ insécable\ espace\ demi-cadratin\ espace\ cadratin}\strut \\
\texttt{numéro\ =\ Nº\ qui\ est\ comme\ un\ o\ exposant,\ différent\ de\ degré\ °}\strut \\
\texttt{degré\ °\ et\ aussi\ ℃\ ℉\ K}\strut \\
\texttt{\#\ †\ ‡\ ‘\ ’\ •\ §\ ™\ ©\ ®\ µ\ ¶}\strut \\
\texttt{\%\ @\ +\ -\ *\ /\ \textbackslash{}\ \textless{}\ \textgreater{}\ =\ ²\ ×\ ÷\ ¼\ ½\ ¾\ ‰\ \_\ ¤\ \$\ £\ ƒ}\strut \\
\texttt{↉\ ⅟\ ½\ ⅓\ ¼\ ⅕\ ⅙\ ⅐\ ⅛\ ⅑\ ⅒\ ⅔\ ⅖\ ¾\ ⅗\ ⅜\ ⅘\ ⅚\ ⅝\ ⅞\ \%\ ‰\ ‱}\strut \\
\texttt{“”\ «»\ ‘’\ ‚‛\ ‹›}\strut \\
Cf. \href{https://fr.wikipedia.org/wiki/Ponctuation\#En_fran\%C3\%A7ais}{Wikipedia Ponctuation en français} Cf. \href{http://david.carella.free.fr/fr/typographie/caracteres-et-symboles-unicode.html}{Caractères et symboles Unicode}

\hypertarget{indicateur-ordinal}{%
\subsection{Indicateur ordinal}\label{indicateur-ordinal}}

\begin{itemize}
\tightlist
\item
  1er 2e 2d 2de 1ers 1res 2es 2ds
\item
  \texttt{U+00BA\ º} MASCULINE ORDINAL INDICATOR (º) \texttt{1º\ 2º\ 3º}
\item
  \texttt{U+00AA\ ª} FEMININE ORDINAL INDICATOR (ª)
\item
  Different from

  \begin{itemize}
  \tightlist
  \item
    \texttt{U+00B0\ °} DEGREE SIGN
  \item
    \texttt{U+02DA\ ˚} RING ABOVE
  \item
    \texttt{U+1D52\ ᵒ} MODIFIER LETTER SMALL O
  \item
    \texttt{U+1D3C\ ᴼ} MODIFIER LETTER CAPITAL O
  \item
    \texttt{U+2070\ ⁰} SUPERSCRIPT ZERO
  \item
    \texttt{U+1D43\ ᵃ} MODIFIER LETTER SMALL A
  \end{itemize}
\end{itemize}

\hypertarget{mathuxe9matiques}{%
\subsection{Mathématiques}\label{mathuxe9matiques}}

\hypertarget{symboles-courants}{%
\subsubsection{Symboles courants}\label{symboles-courants}}

\begin{itemize}
\tightlist
\item
  `+ - / * =
\item
  \texttt{+\ ‒\ ÷\ ×\ =\ ¬\ ≊\ ≅\ ≈\ \textgreater{}\ ≥\ \textless{}\ ≤\ ∞\ ∑\ ∏\ ≠\ √\ ∛\ ∜\ 𝜋\ 𝛑\ ±\ ǁ}
\item
  \texttt{−\ (U+2212\ MINUS\ SIGN)} ne pas confondre avec \texttt{‒\ (U+2012\ FIGURE\ DASH/TIRET\ NUMÉRIQUE)}
\item
  \texttt{≡\ ≢\ ⩵\ ⩶\ ∥\ ∦\ ∟\ ⊾\ ∠\ ∡\ ∢\ ⟀}
\item
  \texttt{⊕\ (XOR)\ ⇒\ ⇔}
\item
  \texttt{⁄\ (U+2044\ FRACTION\ BAR)\ ⌖\ ∧\ ∨}
\item
  \texttt{∩\ ∪\ ∋\ ∌\ ∅\ ∖\ ∄\ ∃\ ∈\ ∉\ ∀\ ⊂\ ⊃\ ⊄\ ⊅}
\item
  \texttt{⁰\ ¹\ ²\ ³\ ⁴\ ⁵\ ⁶\ ⁷\ ⁸\ ⁹\ ⁿ}
\item
  \texttt{ℕ\ ℤ\ ⅅ\ ℚ\ ℝ\ ℂ}
\item
  \texttt{ℇ\ ᵠ\ 𝜑\ ⅈ\ ∞\ 𝞅}
\item
  \texttt{∏\ ∑\ 𝛌\ 𝜆\ 𝝀\ 𝓍\ 𝒳\ 𝑓\ 𝑦\ ∫\ ∬\ ∭\ ε\ \textbar{}}
\item
  \texttt{𝑦\ =\ 𝑓(𝓍)}
\item
  \texttt{|𝓍|\ U+FF5C}
\end{itemize}

\hypertarget{codes-unicode}{%
\subsubsection{Codes Unicode}\label{codes-unicode}}

\begin{itemize}
\tightlist
\item
  \texttt{×\ U+00D7} multiply
\item
  \texttt{÷\ U+00F7} divide
\item
  \texttt{−\ U+2212} minus sign
\item
  \texttt{∗\ U+2217} asterisk operator
\item
  \texttt{⁄\ U+2044} barre de fraction
\item
  \texttt{≈\ U+2248} Presque égal à
\item
  \texttt{≃\ U+2243} Asymptotiquement égal à
\item
  \texttt{𝜋\ U+1D70B} \href{https://en.wikipedia.org/wiki/Pi_(letter)}{Mathematical Italic Small Pi}
\item
  \texttt{𝛑\ U+1D6D1} Mathematical Bold Small Pi
\item
  \texttt{π\ U+03C0}
\item
  \texttt{ℇ\ U+2107} Euler
\item
  \texttt{ⅈ\ U+2148}
\item
  \texttt{φ\ U+03D5} \href{https://en.wikipedia.org/wiki/Phi}{Greek Phi Symbol}
\item
  \texttt{∞\ U+221E}
\end{itemize}

Cf. \href{https://www.compart.com/fr/unicode/category/Sm}{Unicode Symbole mathématique}

\hypertarget{chiffres-romains}{%
\subsubsection{Chiffres romains}\label{chiffres-romains}}

\begin{longtable}[]{@{}llllllllllllll@{}}
\toprule()
1 & 2 & 3 & 4 & 5 & 6 & 7 & 8 & 9 & 10 & 11 & 12 & 50 & 100 \\
\midrule()
\endhead
Ⅰ & Ⅱ & Ⅲ & Ⅳ & Ⅴ & Ⅵ & Ⅶ & Ⅷ & Ⅸ & Ⅹ & Ⅺ & Ⅻ & Ⅼ & Ⅽ \\
ⅰ & ⅱ & ⅲ & ⅳ & ⅴ & ⅵ & ⅶ & ⅷ & ⅸ & ⅹ & ⅺ & ⅻ & ⅼ & ⅽ \\
\bottomrule()
\end{longtable}

\begin{longtable}[]{@{}llllll@{}}
\toprule()
500 & 1\,000 & 5\,000 & 10\,000 & 50\,000 & 100\,000 \\
\midrule()
\endhead
Ⅾ ⅠↃ & Ⅿ ⅭⅠↃ ↀ & ↁ ⅠↃↃ V̅ & ↂ ⅭⅭⅠↃↃ X̅ & ↇ ⅠↃↃↃ L̅ & ↈ ⅭⅭⅭⅠↃↃↃ C̅ \\
ⅾ ⅰↄ & ⅿ ⅽⅰↄ ↀ & ↁ ⅰↄↄ v̅ & ↂ ⅽⅽⅰↄↄ x̅ & ↇ ⅰↄↄↄ l̅ & ↈ ⅽⅽⅽⅰↄↄↄ c̅ \\
\bottomrule()
\end{longtable}

\hypertarget{tirets-et-espaces}{%
\subsection{Tirets et espaces}\label{tirets-et-espaces}}

\hypertarget{tiretstraits-dunion}{%
\subsubsection{Tirets--traits d'union}\label{tiretstraits-dunion}}

\begin{longtable}[]{@{}
  >{\raggedright\arraybackslash}p{(\columnwidth - 6\tabcolsep) * \real{0.4568}}
  >{\raggedright\arraybackslash}p{(\columnwidth - 6\tabcolsep) * \real{0.1111}}
  >{\raggedright\arraybackslash}p{(\columnwidth - 6\tabcolsep) * \real{0.1852}}
  >{\raggedright\arraybackslash}p{(\columnwidth - 6\tabcolsep) * \real{0.2469}}@{}}
\toprule()
\begin{minipage}[b]{\linewidth}\raggedright
Caractère
\end{minipage} & \begin{minipage}[b]{\linewidth}\raggedright
Symbole
\end{minipage} & \begin{minipage}[b]{\linewidth}\raggedright
Unicode
\end{minipage} & \begin{minipage}[b]{\linewidth}\raggedright
HTML
\end{minipage} \\
\midrule()
\endhead
trait d'union / signe moins / tiret & Oo-Oo & U+002D & \&\#x2D; \\
trait d'union conditionnel & Oo-­-Oo & U+00AD & \&\#xAD; \\
trait d'union & Oo‐Oo & U+2010 & \&\#x2010; \\
trait d'union insécable & Oo‑Oo & U+2011 & \&\#x2011; \\
tiret numérique / figure dash & Oo‒Oo & U+2012 & \&\#x2012; \\
tiret demi-cadratin ou tiret moyen & Oo--Oo & U+2013 & \&\#x2013; \\
tiret cadratin ou tiret long & Oo---Oo & U+2014 & \&\#x2014; \\
tiret double cadratin & Oo⸺Oo & U+2E3A & \&\#x2E3A; \\
tiret triple cadratin & Oo⸻Oo & U+2E3B & \&\#x2E3B; \\
barre horizontale & Oo―Oo & U+2015 & \&\#x2015; \\
puce trait d'union & Oo⁃Oo & U+2043 & \&\#x2043; \\
signe moins & Oo−Oo & U+2212 & \&\#x2212; \\
filet horizontal & Oo─Oo & U+2500 & \&\#x2500; \\
filet horizontal double & Oo──Oo & U+2500,U+2500 & \&\#x2500;\&\#x2500; \\
Tiret cadratin minuscule & O﹘Oo & U+FE58 & \&\#xFE58; \\
Tiret minuscule & O﹣Oo & U+FE63 & \&\#xFE63; \\
Tiret pleine chasse & Oo-Oo & U+FF0D & \&\#xFF0D; \\
\bottomrule()
\end{longtable}

Source :

\begin{itemize}
\tightlist
\item
  \href{https://www.compart.com/fr/unicode/category/Pd}{Tiret}
\item
  \href{https://fr.wikipedia.org/wiki/Tiret}{Tiret - Wikipedia}
\item
  \href{https://en.wikipedia.org/wiki/Dash}{Dash = Tiret}
\item
  \href{https://fr.wikipedia.org/wiki/Trait_d\%27union}{Trait d'Union - Wikipedia}
\item
  \href{https://en.wikipedia.org/wiki/Hyphen}{Hyphen = Trait d'Union}
\item
  \href{https://fr.wikipedia.org/wiki/Signes_plus_et_moins}{Signe Plus et Moins}
\item
  \href{https://en.wikipedia.org/wiki/Plus_and_minus_signs}{Plus Minus sign}
\end{itemize}

\hypertarget{trait-dunion-conditionnel-exemple}{%
\paragraph{Trait d'union conditionnel : exemple}\label{trait-dunion-conditionnel-exemple}}

\hypertarget{anticonstitutionnellement}{%
\subparagraph{anticonstitutionnellement}\label{anticonstitutionnellement}}

anticonstitutionnellement anticonstitutionnellement anticonstitutionnellement anticonstitutionnellement anticonstitutionnellement anticonstitutionnellement anticonstitutionnellement anticonstitutionnellement anticonstitutionnellement anticonstitutionnellement anticonstitutionnellement anticonstitutionnellement

\hypertarget{anti-consti-tution-nellemement}{%
\subparagraph{anti-consti-tution-nellemement}\label{anti-consti-tution-nellemement}}

anti­consti­tution­nellement anti­consti­tution­nellement anti­consti­tution­nellement anti­consti­tution­nellement anti­consti­tution­nellement anti­consti­tution­nellement anti­consti­tution­nellement anti­consti­tution­nellement anti­consti­tution­nellement anti­consti­tution­nellement anti­consti­tution­nellement anti­consti­tution­nellement anti­consti­tution­nellement anti­consti­tution­nellement anti­consti­tution­nellement anti­consti­tution­nellement anti­consti­tution­nellement anti­consti­tution­nellement anti­consti­tution­nellement anti­consti­tution­nellement anti­consti­tution­nellement

\hypertarget{a-n-t-i-c-o-n-s-t-i-t-u-t-i-o-n-n-e-l-l-e-m-e-m-e-n-t}{%
\subparagraph{a-n-t-i-c-o-n-s-t-i-t-u-t-i-o-n-n-e-l-l-e-m-e-m-e-n-t}\label{a-n-t-i-c-o-n-s-t-i-t-u-t-i-o-n-n-e-l-l-e-m-e-m-e-n-t}}

a­n­t­i­c­o­n­s­t­i­t­u­t­i­o­n­n­e­l­l­e­m­e­n­t­ a­n­t­i­c­o­n­s­t­i­t­u­t­i­o­n­n­e­l­l­e­m­e­n­t­ a­n­t­i­c­o­n­s­t­i­t­u­t­i­o­n­n­e­l­l­e­m­e­n­t­ a­n­t­i­c­o­n­s­t­i­t­u­t­i­o­n­n­e­l­l­e­m­e­n­t­ a­n­t­i­c­o­n­s­t­i­t­u­t­i­o­n­n­e­l­l­e­m­e­n­t­ a­n­t­i­c­o­n­s­t­i­t­u­t­i­o­n­n­e­l­l­e­m­e­n­t­ a­n­t­i­c­o­n­s­t­i­t­u­t­i­o­n­n­e­l­l­e­m­e­n­t­ a­n­t­i­c­o­n­s­t­i­t­u­t­i­o­n­n­e­l­l­e­m­e­n­t­ a­n­t­i­c­o­n­s­t­i­t­u­t­i­o­n­n­e­l­l­e­m­e­n­t­ a­n­t­i­c­o­n­s­t­i­t­u­t­i­o­n­n­e­l­l­e­m­e­n­t­ a­n­t­i­c­o­n­s­t­i­t­u­t­i­o­n­n­e­l­l­e­m­e­n­t­ a­n­t­i­c­o­n­s­t­i­t­u­t­i­o­n­n­e­l­l­e­m­e­n­t­

\hypertarget{espaces}{%
\subsubsection{Espaces}\label{espaces}}

\begin{longtable}[]{@{}
  >{\raggedright\arraybackslash}p{(\columnwidth - 6\tabcolsep) * \real{0.1236}}
  >{\raggedright\arraybackslash}p{(\columnwidth - 6\tabcolsep) * \real{0.0347}}
  >{\raggedright\arraybackslash}p{(\columnwidth - 6\tabcolsep) * \real{0.0541}}
  >{\raggedright\arraybackslash}p{(\columnwidth - 6\tabcolsep) * \real{0.7876}}@{}}
\toprule()
\begin{minipage}[b]{\linewidth}\raggedright
Caractère
\end{minipage} & \begin{minipage}[b]{\linewidth}\raggedright
Symbole
\end{minipage} & \begin{minipage}[b]{\linewidth}\raggedright
Unicode
\end{minipage} & \begin{minipage}[b]{\linewidth}\raggedright
Description
\end{minipage} \\
\midrule()
\endhead
tabulation & Oo Oo & U+0009 & \\
line feed & ⸻LF & U+000A & \\
line tabulation / vertical tab & ⸻VT & U+000B & \\
form feed & ⸻FF & U+000C & \\
carriage return & ---CR LF & U+000D & \\
CR LF & -CRLF & U+000DU+000A & \\
next line & ⸻NEL & U+0085 & \\
espace & Oo Oo & U+0020 & espace normale, espace sécable, dite aussi «~espace-mot~» \\
espace insécable & Oo~Oo & U+00A0 & no-break space en anglais, pour espace insécable. \\
ogham space mark & Oo Oo & U+1680 & Used for interword separation in Ogham text. Normally a vertical line in vertical text or a horizontal line in horizontal text, but may also be a blank space in ``stemless'' fonts. Requires an Ogham font. \\
en quad & Oo Oo & U+2000 & équivalent à U+2002 \\
em quad & Oo Oo & U+2001 & équivalent à U+2003 \\
espace demi-cadratin & Oo Oo & U+2002 & \\
espace cadratin & Oo Oo & U+2003 & de la largeur d'un M, normalement,~\&\#x2003;~(ou~\&\#8195;) \\
1⁄3 cadratin & Oo Oo & U+2004 & \&emsp13; \\
1⁄4 cadratin & Oo Oo & U+2005 & \&emsp14; \\
1⁄6 de cadratin & Oo Oo & U+2006 & \\
figure space & Oo Oo & U+2007 & Figure space. In fonts with monospaced digits, equal to the width of one digit. HTML/XML named entity: \\
punctuation space & Oo Oo & U+2008 & As wide as the narrow punctuation in a font, i.e.~the advance width of the period or comma.{[}2{]} HTML/XML named entity: \\
espace fine sécable & Oo Oo & U+2009 & thin space en anglais \\
espace ultra-fine & Oo Oo & U+200A & \\
line separator & ⸻ & U+2028 & \\
paragraph separator & ⸻ & U+2029 & \\
espace fine insécable & Oo\,Oo & U+202F & espace insécable étroite dans la traduction française d'Unicode ; Narrow Non Breaking Space \\
medium mathematical space & Oo Oo & U+205F & Used in mathematical formulae. Four-eighteenths of an em. \\
ideographic space & Oo   Oo & U+3000 & As wide as a CJK character cell (fullwidth). Used, for example, in tai tou. \\
espace, symbole 1 & Oo␢Oo & U+2422 & Ce glyphe est une représentation visuelle, utilisée lorsque l'on souhaite matérialiser graphiquement une espace. Nommé en Unicode symbole visuel pour l'espace (ou blank symbol, en anglais). \\
espace, symbole 2 & Oo␣Oo & U+2423 & Ce glyphe est une représentation visuelle, utilisée lorsque l'on souhaite matérialiser graphiquement une espace. Nommé en Unicode boîte ouverte (ou open box, en anglais) \\
\bottomrule()
\end{longtable}

Pour aller plus loin :

\begin{itemize}
\tightlist
\item
  \href{https://www.compart.com/fr/unicode/bidiclass/WS}{Unicode Classe WS}
\item
  \href{https://www.compart.com/fr/unicode/category/Zs}{Unicode Catégorie Zs}
\item
  \href{https://en.wikipedia.org/wiki/Newline\#Unicode}{NewLine}
\item
  \href{https://en.wikipedia.org/wiki/Whitespace_character\#:~:text=In\%20computer\%20programming\%2C\%20whitespace\%20is,an\%20area\%20on\%20a\%20page.}{WhiteSpace Character}
\item
  \href{https://www.compart.com/en/unicode/category/Zs}{Unicode Space Separator}
\item
  \href{https://unicode-explorer.com/b/2000}{Unicode General Punctuation}
\end{itemize}

\hypertarget{symboles}{%
\subsection{Symboles}\label{symboles}}

\texttt{🇫🇷} DRAPEAU FRANÇAIS

\begin{itemize}
\tightlist
\item
  U+1F1EB U+1F1F7 Regional Indicator Symbol Letter F Regional Indicator Symbol Letter R
\item
  \href{https://unicode.org/emoji/charts/full-emoji-list.html}{emoji}
\end{itemize}

\hypertarget{fullwidth}{%
\subsection{Fullwidth}\label{fullwidth}}

\begin{itemize}
\tightlist
\item
  \texttt{➕❌🖊️🔲⚪✅💑}
\item
  \href{https://www.fileformat.info/info/unicode/block/halfwidth_and_fullwidth_forms/list.htm}{Fullwidth Halfwidth}
\end{itemize}

\hypertarget{divers}{%
\subsection{Divers}\label{divers}}

\begin{Shaded}
\begin{Highlighting}[]
\KeywordTok{\textless{}pre\textgreater{}}

\NormalTok{    Char Number Comment}
\NormalTok{    ← }\DecValTok{\&\#x2190;} \DecValTok{\&\#8592;}\NormalTok{ left arrow / APL WIKIPEDIA}
\NormalTok{    ↑ }\DecValTok{\&\#x2191;} \DecValTok{\&\#8593;}\NormalTok{ up arrow}
\NormalTok{    → }\DecValTok{\&\#x2192;} \DecValTok{\&\#8594;}\NormalTok{ right arrow}
\NormalTok{    ↓ }\DecValTok{\&\#x2193;} \DecValTok{\&\#8595;}\NormalTok{ down arrow}

\NormalTok{    ⟵ U+27F5 https://unicode{-}table.com/fr/sets/arrow{-}symbols/}
\NormalTok{    ⟶ U+27F6}
\NormalTok{    ⇒ U+21D2 https://unicode{-}table.com/en/sets/mathematical{-}signs/}
\NormalTok{    ⇐ U+21D0}
\NormalTok{    ⇔ U+21D4}

\NormalTok{    ✓ }\DecValTok{\&\#x2198;} \DecValTok{\&\#8600;}
\NormalTok{    □ }\DecValTok{\&\#x25A1;} \DecValTok{\&\#9633;}
\NormalTok{    ■ }\DecValTok{\&\#x25A0;} \DecValTok{\&\#9632;}

\NormalTok{    ✅ U+2705}
\NormalTok{    ✓ U+2713}
\NormalTok{    ✔ U+2714}
\NormalTok{    ☑ U+2611}
\NormalTok{    🗸 U+1F5F8 }\DecValTok{\&\#128504;}


\NormalTok{    ❌ U+274C }\DecValTok{\&\#10060;}
\NormalTok{    ❎ U+274E }\DecValTok{\&\#10062;}
\NormalTok{    ✖ U+2716}
\NormalTok{    ✗ U+2717}
\NormalTok{    ✘ U+2718}
\NormalTok{    ☒ U+2612}
\NormalTok{    𐄂 U+10102}


\NormalTok{    https://unicode{-}table.com/en/emoji/people{-}and{-}body/}

\NormalTok{    👍 U+1F44D}
\NormalTok{    ♂ U+2642 Male}
\NormalTok{    ♀ U+2640 Female Femelle}
\NormalTok{    ⚥ U+26A5 Male Female}
\NormalTok{    ⚭ U+26AD Mariage}
\NormalTok{    👦 U+1F466 Garçon}
\NormalTok{    👧 U+1F467 Fille}
\NormalTok{    👨 U+1F468 Homme}
\NormalTok{    👩 U+1F469 Femme}
\NormalTok{    🤰 U+1F930 Pregnant Woman Femme enceinte}

\NormalTok{    0 1 2 3 4 5 6 7 8 9 A B C D E F}
\NormalTok{    U+1F600 😀 😁 😂 😃 😄 😅 😆 😇 😈 😉 😊 😋 😌 😍 😎 😏}
\NormalTok{    U+1F610 😐 😑 😒 😓 😔 😕 😖 😗 😘 😙 😚 😛 😜 😝 😞 😟}
\NormalTok{    U+1F620 😠 😡 😢 😣 😤 😥 😦 😧 😨 😩 😪 😫 😬 😭 😮 😯}
\NormalTok{    U+1F630 😰 😱 😲 😳 😴 😵 😶 😷 😸 😹 😺 😻 😼 😽 😾 😿}
\NormalTok{    U+1F640 🙀 🙁 🙂 🙃 🙄 🙅 🙆 🙇 🙈 🙉 🙊 🙋 🙌 🙍 🙎 🙏}
\KeywordTok{\textless{}/pre\textgreater{}}
\end{Highlighting}
\end{Shaded}

\hypertarget{cartes}{%
\subsection{Cartes}\label{cartes}}

\begin{longtable}[]{@{}
  >{\raggedright\arraybackslash}p{(\columnwidth - 32\tabcolsep) * \real{0.1273}}
  >{\raggedright\arraybackslash}p{(\columnwidth - 32\tabcolsep) * \real{0.0545}}
  >{\raggedright\arraybackslash}p{(\columnwidth - 32\tabcolsep) * \real{0.0545}}
  >{\raggedright\arraybackslash}p{(\columnwidth - 32\tabcolsep) * \real{0.0545}}
  >{\raggedright\arraybackslash}p{(\columnwidth - 32\tabcolsep) * \real{0.0545}}
  >{\raggedright\arraybackslash}p{(\columnwidth - 32\tabcolsep) * \real{0.0545}}
  >{\raggedright\arraybackslash}p{(\columnwidth - 32\tabcolsep) * \real{0.0545}}
  >{\raggedright\arraybackslash}p{(\columnwidth - 32\tabcolsep) * \real{0.0545}}
  >{\raggedright\arraybackslash}p{(\columnwidth - 32\tabcolsep) * \real{0.0545}}
  >{\raggedright\arraybackslash}p{(\columnwidth - 32\tabcolsep) * \real{0.0545}}
  >{\raggedright\arraybackslash}p{(\columnwidth - 32\tabcolsep) * \real{0.0545}}
  >{\raggedright\arraybackslash}p{(\columnwidth - 32\tabcolsep) * \real{0.0545}}
  >{\raggedright\arraybackslash}p{(\columnwidth - 32\tabcolsep) * \real{0.0545}}
  >{\raggedright\arraybackslash}p{(\columnwidth - 32\tabcolsep) * \real{0.0545}}
  >{\raggedright\arraybackslash}p{(\columnwidth - 32\tabcolsep) * \real{0.0545}}
  >{\raggedright\arraybackslash}p{(\columnwidth - 32\tabcolsep) * \real{0.0545}}
  >{\raggedright\arraybackslash}p{(\columnwidth - 32\tabcolsep) * \real{0.0545}}@{}}
\toprule()
\begin{minipage}[b]{\linewidth}\raggedright
\end{minipage} & \begin{minipage}[b]{\linewidth}\raggedright
0
\end{minipage} & \begin{minipage}[b]{\linewidth}\raggedright
1
\end{minipage} & \begin{minipage}[b]{\linewidth}\raggedright
2
\end{minipage} & \begin{minipage}[b]{\linewidth}\raggedright
3
\end{minipage} & \begin{minipage}[b]{\linewidth}\raggedright
4
\end{minipage} & \begin{minipage}[b]{\linewidth}\raggedright
5
\end{minipage} & \begin{minipage}[b]{\linewidth}\raggedright
6
\end{minipage} & \begin{minipage}[b]{\linewidth}\raggedright
7
\end{minipage} & \begin{minipage}[b]{\linewidth}\raggedright
8
\end{minipage} & \begin{minipage}[b]{\linewidth}\raggedright
9
\end{minipage} & \begin{minipage}[b]{\linewidth}\raggedright
A
\end{minipage} & \begin{minipage}[b]{\linewidth}\raggedright
B
\end{minipage} & \begin{minipage}[b]{\linewidth}\raggedright
C
\end{minipage} & \begin{minipage}[b]{\linewidth}\raggedright
D
\end{minipage} & \begin{minipage}[b]{\linewidth}\raggedright
E
\end{minipage} & \begin{minipage}[b]{\linewidth}\raggedright
F
\end{minipage} \\
\midrule()
\endhead
U+1F0Ax & 🂠 & 🂡 & 🂢 & 🂣 & 🂤 & 🂥 & 🂦 & 🂧 & 🂨 & 🂩 & 🂪 & 🂫 & 🂬 & 🂭 & 🂮 & \\
U+1F0Bx & & 🂱 & 🂲 & 🂳 & 🂴 & 🂵 & 🂶 & 🂷 & 🂸 & 🂹 & 🂺 & 🂻 & 🂼 & 🂽 & 🂾 & 🂿 \\
U+1F0Cx & & 🃁 & 🃂 & 🃃 & 🃄 & 🃅 & 🃆 & 🃇 & 🃈 & 🃉 & 🃊 & 🃋 & 🃌 & 🃍 & 🃎 & 🃏 \\
U+1F0Dx & & 🃑 & 🃒 & 🃓 & 🃔 & 🃕 & 🃖 & 🃗 & 🃘 & 🃙 & 🃚 & 🃛 & 🃜 & 🃝 & 🃞 & 🃟 \\
U+1F0Ex & 🃠 & 🃡 & 🃢 & 🃣 & 🃤 & 🃥 & 🃦 & 🃧 & 🃨 & 🃩 & 🃪 & 🃫 & 🃬 & 🃭 & 🃮 & 🃯 \\
U+1F0Fx & 🃰 & 🃱 & 🃲 & 🃳 & & 🃴 & 🃵 & & & & & & & & & \\
\bottomrule()
\end{longtable}

\begin{Shaded}
\begin{Highlighting}[]
\NormalTok{    ♥ U+2665 ♡ U+2661}
\NormalTok{    ♠ U+2660 ♤ U+2664}
\NormalTok{    ♦ U+2666 ♢ U+2662}
\NormalTok{    ♣ U+2663 ♧ U+2667}
\end{Highlighting}
\end{Shaded}

\hypertarget{sources}{%
\subsection{Sources}\label{sources}}

(Compart.com){[}https://www.compart.com/fr/unicode/{]}

\backmatter
\end{document}
